\documentclass{article}
\usepackage[margin=1in]{geometry}
\usepackage{amsmath,amsthm,amsfonts,amssymb,amscd}
\usepackage{circuitikz}
\usepackage{mdframed}
\usepackage{titlesec}
\usepackage{chemfig}
\usepackage{minted}
\usepackage{float}
\usepackage{titlesec}
\usepackage{graphicx}
\usepackage{ifthen}
\usepackage{fancyhdr}
\usepackage{musixtex}



% -------- %
% SECTIONS %
% -------- %
\newcounter{problemnumber}\setcounter{problemnumber}{1}
\titlespacing\section{0pt}{10pt}{0pt}   % Spacing between Problems
\titlespacing\subsection{0pt}{5pt}{0pt} % Spacing between Parts
\newcommand{\problem}[1][-1]{
	\setcounter{partnumber}{1}
	\ifnum#1>0
		\setcounter{problemnumber}{#1}
	\fi
	\section*{Problem \arabic{problemnumber}}
	\stepcounter{problemnumber}
}

\newcounter{partnumber}\setcounter{partnumber}{1}
\newcommand{\ppart}[1][-1]{
	\ifnum#1>0
		\setcounter{partnumber}{#1}
	\fi
	\subsection*{Part \parttype{partnumber}}
	\stepcounter{partnumber}
}

\newenvironment{question}{
	\color{gray}\itshape
	
	\vspace{5pt}
	\begin{tabular}{|p{0.97\linewidth}}
}{
	\end{tabular}\\[5pt]
}



% ------------- %
% HEADER/FOOTER %
% ------------- %
\setlength{\parindent}{0pt}
\setlength{\headheight}{30pt}
\headsep=0.25in
\pagestyle{fancy}
\lhead{\ifthenelse{\thepage=1}
	{\textbf{Jack Leightcap} \\ \textbf{\writeday}}
}
\chead{\ifthenelse{\thepage=1}
	{\textbf{\huge{HOMEWORK \hwnumber}}}
	{\textbf{\large{HOMEWORK \hwnumber}}}
}
\rhead{\ifthenelse{\thepage=1}
	{\textbf{{\course}} \\ \textbf{Professor {\prof}}}
}
\cfoot{\thepage}
\renewcommand\headrulewidth{0.4pt}
\renewcommand\footrulewidth{0.4pt}



% ---------- %
% PARAMETERS %
% ---------- %
% \PARTTYPE:
% \Alph   := "Part A, Part B,  ..."
% \alph   := "Part a, Part b,  ..."
% \arabic := "Part 1, Part 2,  ..."
% \Roman  := "Part I, Part II, ..."
\newcommand\parttype{\Roman}

% \COURSE:
\newcommand\course{EECE 1234}

% \HWNUMBER
\newcommand\hwnumber{1}

% \SEMESTER
\newcommand\semester{Fall 2019}

% \PROF
\newcommand\prof{Schmitty}

% \WRITEDAY
% \today is date of compilation, replace if writing due date rather than write date
\newcommand\writeday{\today}



%  ------- %
% DOCUMENT %
% -------- %
\begin{document}
\problem
Problem quotes use \emph{quesiton} environment.
\begin{question}	
	Problem statements in italics using the \emph{question} environment.
	Since wrapped in a \emph{tabular} environment, inline math like \(a^2 + b^2 = c^2\) is allowed. 
\end{question}
Example \texttt{chemfig},
\begin{figure}[H]
	\centering
	\chemfig{C(-[5]H)(-[2]H)(<[:-70]H)(<:[:-20]H)}
	\hspace{2cm}
	\chemfig{H_3C-CH(-[2]CH_3)-CH_3}
\end{figure}
\begin{figure}[H]
	\centering
	\chemnameinit{\chemfig{R-C(-[:-30]OH)=[:30]O}}
	\schemestart
		\chemname{\chemfig{R'OH}}{Alcohol}
		\+
		\chemname{\chemfig{R-C(-[:-30]OH)=[:30]O}}{Carboxylic acid}
		\arrow(.mid east--.mid west)
		\chemname{\chemfig{R-C(-[:-30]OR')=[:30]O}}{Ester}
		\+
		\chemname{\chemfig{H_2O}}{Water}
	\schemestop
	\chemnameinit{}
\end{figure}
\problem
Syntax highlighting code blocks is simple if using \texttt{minted}, however requires eternal python compilation.
This makes very pretty code blocks that are ulikely to compile on another setup without fidling.
\ppart
\begin{minted}{python}
def fib(n):
    if n <= 1:
        return n
    else:
        return fib(n-1) + fib(n-2)
\end{minted}
\ppart
\begin{minted}{racket}
(define (fib n)
    (cond [(<= n 1) n]
          [else (+ (fib (- n 1)) (fib (- n 2)))]))
\end{minted}
\problem
Circuit diagrams with \texttt{circuitikz},
\begin{figure}[H]
	\centering
	\begin{circuitikz}
        	\draw (0,0)
		to[V,v= $V_\text{ext}(t)$ ] (0,2)
                to[short] (2,2)
                to[R= $R_1$ ] (2,0)
                to[short] (0,0);
                \draw (2,2)
                to[short] (4,2)
                to[L= $L_1$ ] (4,0)
                to[short] (2,0);
                \draw (4,2)
                to[short] (6,2)
                to[C= $C_1$ ] (6,0)
                to[short] (4,0);
        \end{circuitikz}
\end{figure}
\problem
Always box final answers with \texttt{{\textbackslash}boxed\{\}},
\[ \boxed{\int \nabla \cdot \textbf{A} d\tau = \oint \textbf{A} \cdot d\textbf{a}}\]
\problem
Header changes past first page.
\problem
Music notation using \texttt{musixtex}.
In practice, I'd probably use \texttt{lilypond} and import as an image.\\
\begin{music}
	\parindent10mm
	\instrumentnumber{1}
	\setname1{Piano}
	\setstaffs1{2}
	\generalmeter{\meterfrac44}
	\startextract
		\Notes\ibu0f0\qb0{cge}\tbu0\qb0g|\hl j\en
		\Notes\ibu0f0\qb0{cge}\tbu0\qb0g|\ql l\sk\ql n\en\bar
		\Notes\ibu0f0\qb0{dgf}|\qlp i\en
		\Notes\tbu0\qb0g|\ibbl1j3\qb1j\tbl1\qb1k\en
		\Notes\ibu0f0\qb0{cge}\tbu0\qb0g|\hl j\en
  	\zendextract
\end{music}
However, \texttt{musixtex} allows direct integration in a \texttt{*.tex} file, compared to something like \texttt{*.lytex}.
\problem[10]
Problem number skips with \texttt{{\textbackslash}problem[\#]}. \texttt{{\textbackslash}problem} without an argument defaults to counting up (as most assignments have strictly increasing question numbers), so next problem is...
\problem
\end{document}
